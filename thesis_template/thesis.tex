%%%%%%%%%%%%%%%%%%%%%%%%%%%%%%%%%%%%%%%%%%%%%%%%%%%%%%%%%%%
% EPFL report package, main thesis file Goal: provide formatting for theses and
% project reports Author: Mathias Payer <mathias.payer@epfl.ch>
%
% This work may be distributed and/or modified under the conditions of the
% LaTeX Project Public License, either version 1.3 of this license or (at your
% option) any later version.  The latest version of this license is in
% http://www.latex-project.org/lppl.txt
%
%%%%%%%%%%%%%%%%%%%%%%%%%%%%%%%%%%%%%%%%%%%%%%%%%%%%%%%%%%%


% useful links:
% REPICA: https://ieeexplore.ieee.org/stamp/stamp.jsp?arnumber=8454360
%         good state of the art overview, both ARM64 and arm32
%         1. do not use symbolization, but relative address correction
%         2. Detect jump tables and global pointers, but not fix them
%         3. Overhead is exactly the same as retrowrite


%\pdfcompresslevel=0 \pdfobjcompresslevel=0

\documentclass[a4paper,11pt,oneside]{report}
\usepackage[MScThesis,lablogo]{EPFLreport} \usepackage{xspace}
\usepackage{xcolor} \usepackage{soul}


\title{Arm Wrestling: porting the Retrowrite project to the ARM architecture}
\author{Luca Di Bartolomeo} \adviser{Prof. Mathias Payer (EPFL)}
\supervisor{Prof. Kenny Paterson (ETH)}

\linespread{1.5} 
\newcommand{\sysname}{Retrowrite\xspace}

%\definecolor{orange}{rgb}{1,0.5,0}
%\newcommand{\todo}[1]{\colorbox{cyan}{\parbox{0.9\textwidth}{#1}}}
\newcommand{\todo}[1]{%
	\begingroup 
	\sethlcolor{cyan}%
	\hl{TODO: #1}%
	\endgroup
}


\dedication{ 
\begin{raggedleft}
	No matter where you go, everyone is connected.\\
	--- Serial Experiments Lain\\
\end{raggedleft} 
\vspace{4cm} 
\begin{center}
	Dedicated to my parents, my sister Sara, my dear Giulia, to my friends back 
	in Rome and to my roommates Matteo and Filippo who all inspired me and kept 
	up with my constant complaining. Thanks!
\end{center} 
}

\acknowledgments{
	I would like to thank my advisor, Prof. Mathias Payer, for his support,
	guidance, and for trusting me by assigning me this inspiring project.  I
	admire him a lot and I wish all the best for him, and in particular I hope
	that I can work on many other projects with him.
	
	I would also like to thank his research group, HexHive, as I always found
	myself very welcome there, even if I could visit them once a week. Those
	have been six very happy months in which I learned quite a lot of things,
	and I have to thank prof. Payer and his doctorate students and researches
	for it.  I wish them all to have a very succesfull career, and I hope we
	can continue to work together in the future!
	
	Special thanks goes to my family and my friends in Rome. Their support was
	always available, and it has always been a huge pleasure to visit them once
	in a while in Italy. I also need to thank my S.O. Giulia, I felt she was
	always behind my back, keeping a good check on my mental sanity during the
	worst times of the outbreak. My roommates too, Matteo and Filippo, deserve
	a mention here, as their patience and their rubber duck debugging skills
	proved to be fundamental during some nasty debugging sessions.
}

\begin{document} \maketitle \makededication \makeacks


\begin{abstract}

	While there were good recent attemps, (\todo{link to PinePhone, System76,
	and similar projects}) using only open-source software is particularly
	hard, and especially on the mobile phone market even the most determined
	users are often forced to use closed-source ARM libraries or modules.  
	Those often run at privileges higher than we might want (e.g. manufacturer
	specific kernel modules, \todo{link to some}), and are also hard to audit
	for their vulnerabilities. 

	Many existing tools were developed to improve the auditability of closed
	source programs, especially aimed at helping the fuzzing process, with
	approaches such as implementing AddressSanitizer (a compiler pass only
	available with the source code) through dynamic instrumentation, but even
	state-of-the-art instrumentation engines incur in prohibitive runtime
	overhead (10x and more). 

	In this thesis, we would like to show that static instrumentation for ARM 
	binaries is a viable alternative to existing approaches, that has less 
	flexibility (only works on C, position independent binaries) but has
	negligible overhead compared to compiled source code. We present the ARM
	port of \texttt{Retrowrite}, an existing static binary rewriter for x86
	executables, which implements the symbolization engine and the memory
	sanitization instrumentation. 

	While \texttt{Retrowrite} is not the only binary rewriter for x86, there 
	are very few that are aimed at the ARM64 architecture. In particular, most 
	of them resort to lifting to an intermediate IR (achieving more flexibility 
	but losing fine-grained control over the instrumented instructions) or use 
	approaches like trampolines which make the whole problem much easier but 
	also introduce a noticeable overhead.

	\texttt{Retrowrite} 

\end{abstract}


\maketoc

%%%%%%%%%%%%%%%%%%%%%%
\chapter{Introduction}
%%%%%%%%%%%%%%%%%%%%%%
%The introduction is a longer writeup that gently eases the reader into your
%thesis~\cite{dinesh20oakland}. Use the first paragraph to discuss the setting.
%In the second paragraph you can introduce the main challenge that you see.
%The third paragraph lists why related work is insufficient.
%The fourth and fifth paragraphs discuss your approach and why it is needed.
%The sixth paragraph will introduce your thesis statement. Think how you can
%distill the essence of your thesis into a single sentence.
%The seventh paragraph will highlight some of your results
%The eights paragraph discusses your core contribution.
%This section is usually 3-5 pages.

\todo{}

%%%%%%%%%%%%%%%%%%%%
\chapter{Background}
%%%%%%%%%%%%%%%%%%%%
%The background section introduces the necessary background to understand your
%work. This is not necessarily related work but technologies and dependencies
%that must be resolved to understand your design and implementation.
%This section is usually 3-5 pages.

\todo{}


%%%%%%%%%%%%%%%%
\chapter{Design}
%%%%%%%%%%%%%%%%
%Introduce and discuss the design decisions that you made during this project.
%Highlight why individual decisions are important and/or necessary. Discuss
%how the design fits together.
%This section is usually 5-10 pages.

\todo{}


%%%%%%%%%%%%%%%%%%%%%%%%
\chapter{Implementation}
%%%%%%%%%%%%%%%%%%%%%%%%
%The implementation covers some of the implementation details of your project.
%This is not intended to be a low level description of every line of code that
%you wrote but covers the implementation aspects of the projects.

%This section is usually 3-5 pages.

\todo{}

%%%%%%%%%%%%%%%%%%%%
\chapter{Evaluation}
%%%%%%%%%%%%%%%%%%%%
%In the evaluation you convince the reader that your design works as intended.
%Describe the evaluation setup, the designed experiments, and how the
%experiments showcase the individual points you want to prove.

%This section is usually 5-10 pages.

\todo{}


%%%%%%%%%%%%%%%%%%%%%%
\chapter{Related Work}
%%%%%%%%%%%%%%%%%%%%%%

%The related work section covers closely related work. Here you can highlight
%the related work, how it solved the problem, and why it solved a different
%problem. Do not play down the importance of related work, all of these
%systems have been published and evaluated! Say what is different and how
%you overcome some of the weaknesses of related work by discussing the 
%trade-offs. Stay positive!
%This section is usually 3-5 pages.


% useful links:
% REPICA: https://ieeexplore.ieee.org/stamp/stamp.jsp?arnumber=8454360
%         good state of the art overview, both ARM64 and arm32
%         1. do not use symbolization, but relative address correction
%         2. Detect jump tables and global pointers, but not fix them
%         3. Overhead is exactly the same as retrowrite
% A SURVEY ON BINARY REWRITING: https://publications.sba-research.org/publications/201906%20-%20GMerzdovnik%20-%20From%20hack%20to%20elaborate%20technique.pdf

{

\setlength{\parindent}{0cm}

\todo{This section now has only list of related work I would like to talk 
about, will be expanded later}



\textbf{Dynamic rewriters in general}:\\
First, the good ol' ones: \texttt{PIN}\cite{pin}, 
\texttt{Valgrind}\cite{valgrind} and \texttt{DynInst}\cite{dyninst}.

More recently, \texttt{Multiverse} \cite{multiverse}, \texttt{Frida} and 
\texttt{DynamorIO} are interesting examples.

\texttt{QASAN}\cite{qasan} is similar to retrowrite's basan, much slower but 
also much more portable (works on any binary supported by qemu)



\textbf{Static rewriters that use lifting to IR}:\\
\texttt{McSema} \cite{mcsema} is a great example of LLVM IR lifting approach, 
also particularly nice as it supports C++ exceptions

\textbf{Static rewriters that use trampolines}:\\
\texttt{E9patch}\cite{e9patch} uses only trampolines, no need to recover 
control flow, but also noticeable overhead

\textbf{Static rewriters that use symbolization}:\\
\texttt{Uroboros}\cite{uroboros} and \texttt{Ramblr}\cite{ramblr} are the first 
reassembable assembly approaches (symbolization), \texttt{Retrowrite} 
\cite{dinesh20oakland} (x86 version)

\textbf{Static rewriters aimed at ARM binaries}:\\
\texttt{Repica} \cite{repica} is probably the most recent addition to binary 
rewriters aimed at ARM binaries.  

\todo{Maybe add Bistro ? } \cite{bistro}

For more check out recent surveys on the area of binary rewriting 
\cite{binaryrewritingsurvey}

}

%%%%%%%%%%%%%%%%%%%%%
\chapter{Future Work}
%%%%%%%%%%%%%%%%%%%%%

{

\setlength{\parindent}{0cm}
\hangindent=0.7cm \textbf{Support for more source languages}: For now, 
\texttt{Retrowrite} supports only binaries compiled from the C language, both 
for the x86 and the ARM implementation. The easiest addition would be to add 
support for C++ by expanding the analysis capabilities of \texttt{Retrowrite} 
to support exception tables too, but many more languages could be supported in 
the future. 

\hangindent=0.7cm \textbf{Support for kernel space binaries}: Right now the ARM 
port of \texttt{Retrowrite} supports only userspace binaries, contrary to the 
x86 version that supports linux kernel modules too. The kernel version of 
\texttt{Retrowrite\_ARM} would prove to be particularly interesting as it would 
open new ways to efficiently fuzz Android kernel modules.

\hangindent=0.7cm \textbf{Support for more executable formats/operating 
systems}: The current implementation of the \texttt{Retrowrite} tool is aimed 
only towards ELF files, but adding support for MACH-O and PE binaries should 
not require too much effort. This would also be interesting as Windows and 
MacOS present way more closed-source modules compared to Linux.

\hangindent=0.7cm \textbf{More instrumentation passes}: While right now we 
implemented only the AddressSanitizer instrumentation in the ARM port of
\texttt{Retrowrite}, the design of \texttt{Retrowrite} is modular to make
adding new instrumentation passes or new mitigations very easy for anyone. To
name a few, the interesting ones would be:
\begin{itemize}
	\item Shadow stack (return address protection)
	\item ARM pointer authentication (hardware-assisted)
	\item Control flow authentication
	\item Coverage-guidance for fuzzing
\end{itemize}

}

%%%%%%%%%%%%%%%%%%%%
\chapter{Conclusion}
%%%%%%%%%%%%%%%%%%%%

%In the conclusion you repeat the main result and finalize the discussion of
%your project. Mention the core results and why as well as how your system
%advances the status quo.


In summary, we develop the ARM architecture implementation of the 
\texttt{Retrowrite} project, a scalable static rewriter for linux C binaries.  
\texttt{Retrowrite} enables targeted application of static instrumentation 
where no source code is available, such as proprietary binaries, inline 
assembly, or code generated by a deprecated compiler.  We also present an 
example instrumentation pass on top of the symbolization engine, 
AddressSanitizer, particularly useful for fuzzing purposes. We present new 
solutions to problems arising from the peculiarities of the ARM architecture 
such as the fixed-size instruction set, global variable accesses and jump table 
instrumentation. We show that the total overhead of the symbolization and the 
instrumentation pass are competitive with source based AddressSanitizer.  Our 
work shows that \texttt{Retrowrite}'s original approach is not limited to the 
x86 architecture, but can be applied to the ARM architecture and more.



\cleardoublepage
\phantomsection

\addcontentsline{toc}{chapter}{Bibliography}
\printbibliography

% Appendices are optional
% \appendix
% %%%%%%%%%%%%%%%%%%%%%%%%%%%%%%%%%%%%%%
% \chapter{How to make a transmogrifier}
% %%%%%%%%%%%%%%%%%%%%%%%%%%%%%%%%%%%%%%
%
% In case you ever need an (optional) appendix.
%

\end{document}
